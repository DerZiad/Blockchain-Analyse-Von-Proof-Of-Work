\clearpage
\chapter{\textbf{Grundlagen}}\label{grundlagen}
%\addtocontents{toc}{\vspace{0.8cm}}

\section{Unterkapitel}\label{unterkapitel}
\addtocontents{toc}{\vspace{0.8cm}}

Hier folgt ein Beispiel für eine Formel:

% Formel
\begin{equation}\label{waermestrom}
\dot Q = \frac{dQ}{dt} = \lambda \frac{T_1-T_2}{\Delta x} A
\end{equation}

Wie in Gleichung \ref{waermestrom} zu erkennen ist, wird der Wärmestrom $\dot Q$ von der Wärmeleitfähigkeit $\lambda$, der Fläche $A$ und der Temperaturdifferenz $\Delta T = T_1-T_2$ zwischen den betrachteten Orten $\Delta x$ linear beeinflusst.

%% Zwei Abbildungen, die zusammen gehören

%\begin{figure}
%        \centering
%        \begin{minipage}[c]{0.45\textwidth}
%                \includegraphics[height=6.5cm]{pic/dateiname1.png}
%        \end{minipage}
%        \begin{minipage}[c]{0.45\textwidth}
%                \includegraphics[height=6.5cm]{pic/dateiname2.png}
%        \end{minipage}
%        \caption{Zwei Abbildungen}\label{fig:zwei_abb}
%\end{figure}
