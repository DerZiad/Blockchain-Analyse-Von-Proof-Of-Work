% Leslie Lamport hat LaTeX entwickelt, um dem Anwender das Schreiben
% spezieller Dokumente zu vereinfachen. Um den europäischen/deutschen
% Vorstellungen eines Layouts zu entsprechen, hat Markus Kohm analoge
% Dokumentklassen entwickelt (Koma-Klassen)

% Diese Dokumente sind
% im Original    extern              Koma       extern
% 1. article                         scrartcl
% 2. report                          scrreprt
% 3. book                            scrbook
% 4. letter                          scrlttr2
% 5. slide                                      beamer
% 6.             poster                         tikzposter
% 7.                                            komacv bzw. europecv
% Das Layout ist im Original amerikanischen Vorstellungen ensprechend!

% Der Aufbau ist immer

% Kopf des Dokumentes
% ===================
\documentclass[ngerman,               % in eckigen Klammern stehen
                                      % optionale Angaben, hier: ngerman
                                      % - es werden Trennmuster nach
                                      %   _n_euer deutscher Recht-
                                      %   schreibung geladen
                                      % - wenn das Zusatzpaket babel
                                      %   geladen ist, kann der Autor
                                      %   länderspezifische
                                      %   Befehle, beispielsweise für
                                      %   Anführungszeiche direkt
                                      %   eingeben
                                      % typographische Regeln:
                                      % in der scrreprt-Klasse beginnt
                                      % jedes Kapitel auf einer
                                      % neuen Seite, leere Seiten wie
                                      % die Zusammenfassung
                                      % erhalten keine Seitennummer,
                                      % ebensowenig Titelseiten, es
                                      % werden alle(!!!) Seiten intern
                                      % nummeriert, die Titelseite
                                      % hat die Nummer 1, die
                                      % Seitennummerierung erfolgt
                                      % unten mittig
               a4paper,               % Ausgabe auf DIN A4 Seiten
             % draft,                 % Satzspiegelfehler werden 
                                      % angezeigt, Abbildungen werden
                                      % nicht ausgegeben
               fleqn,                 % math. Formeln werden mit festem
                                      % Einzug von links dargestellt
                     ]{scrreprt}

% Einstellungen
% =============

% Da die Einstellungen sich von Dokument zu Dokument kaum verändern,
% ist es sinnvoll, diese an einer zentralen Stelle abzulegen und von
% dort zu laden. Dies geschieht völlig transparent mit dem Befehl
% \input{<Pfad>}, dem als Argument eine Datei mit (relativem) Zugriffs-
% pfad übergeben wird
\input{../../shared/twp-cfg}
% in dieser Datei werden alle Zusatzpakete mittels
% \usepackage{<package>} geladen


% merke: ein späterer Titel überschreibt einen früher gesetzten Titel,
% in der scrreprt-Dokumentenklasse erscheint der Titel auf einer
% eigenen Seite
\title{Ein erstes \LaTeX-Dokument mit der Dokumentklasse \texttt{scrreprt}}       % \texttt{<Inhalt>} formatiert <Inhalt>
                         % mit einer dicktengleichen Schriftart
                         % im Vergleich zur ansonsten eingesetzten
                         % Proportionalschrift

% Rumpf des Dokumentes
\begin{document}
  \maketitle                          % die automatisch erzeugte
                                      % Titelseite, dies erfordert
                                      % die Angabe des Autors und des
                                      % Titels in den Einstellungen

  % Die Zusammenfassung erscheint auf einer eigenen Seite
  \begin{abstract}                    % die abstract-Umgebung gibt es
                                      % nur in der scrartcl-Klasse!
    Hier steht eine kurze Zusammenfassung des Inhaltes, es geht hier um den Aufbau eines Artikels, insbesondere um die geeignete Gliederung.
  \end{abstract}

  \tableofcontents

  % die Gliederung erfolgt in bis zu 7 Ebenen, abhängig von der Klasse
  %   Ebene     Klasse                   Befehl (einfachen Form)
  %   0.        scrbook/scrreprt         \chapter{<Titel>}
  %   1.        alle                     \section{<Titel>}
  %   2.        alle                     \subsection{<Titel>}
  %   3.        alle                     \subsubsection{<Titel>}
  %   4.        alle                     \paragraph{<Titel>}
  %   5.        alle                     \subparagraph{<Titel>}
  %   6.        alle, nicht im Original  \subsubparagraph{<Titel>}

  \chapter{Einleitung}   % Einleitung, Hauptteil und Schluss
                         % dürfen niemals in der fertigen Arbeit
                         % stehen, da muss die Autorin bzw. der Autor
                         % sich etwas Passendes einfallen lassen.

  Hier steht der Inhalt des Dokumentes, weiteres siehe im n\"achsten
  Beispiel. Der erste Absatz wird ohne Einzug formatiert, wenn er einer \"Uberschrift folgt.

  Weitere Absätze bekommen einen Einzug, wie hier deutlich zu sehen ist.
  Das soll an dieser Stelle reichen.

  In dieser Arbeit wird gezeigt, wie man mit \LaTeX{} arbeitet. Dazu
  wird in \cref{chap:TeX} auf die Struktur eines wissenschaftlichen
  Dokumentes eingegangen, dabei wird ein spezieller Augenmerk auf die
  Befehlsstruktur von \LaTeX{} gerichtet (\cref{sec:befehlsstruktur}).


  \chapter{Arbeiten mit \TeX/\LaTeX}% durch dieses %-Zeichen wird
  \label{chap:TeX}                  % Leerraum vermeiden, der zu einer
                                    % falschen Nummerierung führen kann
                                    % \label: setzt eine Marke,
                                    % auf die ich mich an andere Stelle
                                    % beziehen kann.

  In dieser kleinen Einführung wird beschrieben, wie man mit \LaTeX{} arbeitet.

  % eine Kurzform der Überschrift, die für das Inhaltsverzeichnis
  % geeignet ist erzeugt man mit einem optionalen Argument
  \section[Die Befehlsstruktur]{Die Befehlsstruktur von \TeX/\LaTeX}%
  \label{sec:befehlsstruktur}

  % die Nummerierung wird automatisch korekt erzeugt, wenn die
  % Gliederung korrekt ist, eine fehlerhafte Gliederung wird durch 0.
  % in der Überschrift angezeigt. Für die Hausaufgaben ist eine
  % fehlerhafte Nummerierung ein Ausschlusskriterium
  \subsection{\TeX: Einfache Befehle -- Primitive} % das doppelte
             % Minuszeichen
             % ist der sogenannte n-dash, es wird immer dann eingesetzt,
             % wenn ein _deutscher_ Gedankenstrich gebraucht wird

Jedes Zeichen ist ein einfacher Befehl: setze das Zeichen in der
aktuellen Schriftart und der aktuellen Größe an die aktuelle
Position, verändere die aktuelle Position um die Breite des Zeichens.
Es gibt aber Zeichen, die eine besondere Bedeutung haben. Wenn man
diese im Text benötigt, muss man eine Ersatzdarstellung benutzen. Im
Regelfall ist es ein vorweggestellter Gegen-/Rückwärtsschrägstrich
(backslash)
\begin{itemize} % Listenumgebungen werden später behandelt
  \item Das \%-Zeichen leitet Kommentar ein.
  \item Das \$-Zeichen leitet in \TeX ein einfachen Mathematikmodus ein
  und aus.
  \item Das \(\backslash\)-Zeichen leitet Befehle ein, der Befehl endet
   mit dem ersten Nichtbuchstaben. Wird nach einem Befehl ein Leerzeichen benötigt, muss man dieses entweder durch einen weiteres
   \(\backslash\)-Zeichen gefolgt von einem Leerzeichen oder durch ein
   Paar von Gruppenklammern erzeugen: Vergleiche \LaTeX und \LaTeX\
   und \LaTeX{} und auch \LaTeX.
  \item Geschweifte Klammern (\{ und \}) definieren Gruppen. Dinge, die
   in einer Gruppe definiert sind, sind außerhalb der Gruppe unbekannt.
  \item Das Doppelkreuz \# beschreibt Parameter.
  \item Das kaufmännische Und (\&) wird für das Setzen von Tabellen
  gebraucht.
  \item Der Unterstrich (\_) leitet einen Index ein, gilt nur im
   Mathematikmodus.
  \item Das Dach (\^{}) leitet eine Potenz ein, gilt nur im
  Mathematikmodus.
  \item Die Tilde (\~{}) stellt einen Leerraum dar, der nicht umgebrochen werden kann.
\end{itemize}

  \subsection{\LaTeX: Einfache Befehle}

  \LaTeX{} hat die Befehlsstruktur von \TeX{} übernommen, es gibt bei
   vielen Befehlen wie bei den Gliederungsbefehlen optionale Parameter,
   die sich auf die Ausführung/Darstellung auswirken. Ein einfacher
    Befehl hat damit die Syntax
\begin{lstlisting}[language=tex, style=colored]
@\cs{\meta{Kommandoname}}\oarg{opt. Parameter}\marg{verpflichtende Parameter}@
\end{lstlisting}

   Zusätzlich hat \LaTeX{} den Begriff der Umgebung eingeführt: In einer
   Umgebung gibt es zusätzliche Kommandos, die außerhalb der Umgebung
   unbekannt sind. Auch eine Umgebung kann optionale Parameter benutzen,
   diese werden an das einführende Kommando in eckigen Klammern
   angehängt.
\begin{lstlisting}
\begin{@\meta{Umgebungsame}@}@\oarg{Optionen}@
...
\end{@\meta{Umgebungsame}@}
\end{lstlisting}


  \subsection{Gliederung/Strukturierung eines Dokumentes}

  Ein wissenschaftliche Arbeit zeichnet sich durch ein paar Dinge aus:
  \begin{enumerate}
    \item Eine Gliederung entsprechend dem hier vorliegenden Vorbild.

    Diese wird benutzt, um dem Leser den Überblick über den Inhalt und
    das Verständnis des Inhaltes zu erleichtern. Mit einer Gliederung
    kann man den Leser auf wichtige Dinge vorbereiten, man kann auch
    Rückverweise geben.

    Dies geschieht mit den Kommandos
    \lstinline[mathescape]|\label{$\meta{Marke}$}| und
    \lstinline[mathescape]|\cref{$\meta{Marke}$}|. Dabei muss Marke in
    beiden Fällen identisch sein, ein Tippfehler führt zu
    Fehlermeldungen, im Text erscheint \textbf{??}, um den Fehler
    sichtbar zu machen.

  \end{enumerate}



  \subsection{Weitere Untergliederung Teil 4}
  
  Hier wird gezeigt, wie man auf das Inhaltsverzeichnis und die
  Nummerierung der Unterabschnitte Einfluss nehmen kann. Standarmäßig
  wird nummeriert bis zur \lstinline|\subsection|, wenn weitergehend
  nummeriert werden soll, muss man entsprechende Grenzen setzen
  \begin{lstlisting}
    \setcounter{secnumdepth}@\marg{Tiefe}@
\end{lstlisting}
  \meta{Tiefe} gibt die Stufe an, bis zu der nummeriert werden soll.
  
  \subsubsection{Unterunterabschnitt 1}
  
  Kein wichtiger Inhalt

  \subsubsection{Unterunterabschnitt 2}

  Kein wichtiger Inhalt
  
  \paragraph{Unterunterunterabschnitt 1}
  
  Noch weniger wichtig

  \paragraph{Unterunterunterabschnitt 2}

Noch weniger wichtig

  \subparagraph{Unterunterunterunterabschnitt 1}

Noch viel weniger wichtig


  \subparagraph{Unterunterunterunterabschnitt 2}

Noch viel weniger wichtig, wie zuvor

%  \subsubparagraph{Unterunterunterunterunterabschnitt 1}

Noch viel, viel weniger wichtig






  \section[Zweiter Unterabschnitt]{Zweiter Unterabschnitt zu
dem Hauptteil, der in der Zukunft einen wesentlich aussagekräfigeren
Titel bekommen muss}

  \section[Dritter Unterabschnitt]{Dritter Unterabschnitt zu
dem Hauptteil, der in der Zukunft einen wesentlich aussagekräfigeren
Titel bekommen muss}

  \chapter{Schluss}

  In dieser Arbeit wurde in \cref{sec:befehlsstruktur} gezeigt, wie
  \LaTeX-Befehle eingesetzt werden können, um Inhalte zu verdeutlichen.

\end{document}
