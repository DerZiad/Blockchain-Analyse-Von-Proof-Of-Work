\documentclass[version=last, ngerman]{scrlttr2}

% die Konfigurationsdateien...
%23456789012345678901234567890123456789012345678901234567890123456789012
\wlog{This is twp-cfg, the common configuration file all documents
  (JHf)}

% Einstellungen werden in LaTeX vor allen Dingen durch das Einbinden
% von Paketen vorgenommen:
% \usepackage[parameter]{package}

\usepackage{iftex}              % Test des Formatierers, durch \if...
                                % können Anpassungen an den Prozessor
                                % vorgenommen werden
\ifluatex                       % wenn mit dem neuen TeX-Prozessor,
                                % der mehr Fontarten unterstützt,
                                % gearbeitet wird:
                                % - pk-Fonts (Pixelfonts, nicht
                                %   skalierbar, alt)
                                % - pfb-Fonts (Fonts, die durch Kurven
                                %   beschrieben werden, skalierbar,
                                %   entwickelt von Adobe->Acrobat
                                %   Reader)
                                % - otf/ttf-Fonts (Fonts, die durch
                                %   Kurven beschrieben werden,
                                %   skalierbar, entwickelt von Apple
                                %   und Microsoft, um Adobe-Lizenzen
                                %   zu sparen)
  %---- Eingabezeichensatz --------------------------------------------
                                % luatex unterstützt utf-8, also keine
                                % Festlegung des Eingabezeichensatzes
                                % erforderlich
  %---- Grundfont -----------------------------------------------------
  \usepackage{fontspec}         % Festlegen der Fontverwaltung für
                                % LuaTeX.
  \defaultfontfeatures{Renderer=Basic, Ligatures=TeX}
  \fontspec{Latin Modern Roman}
  \setmonofont[Scale=0.85]{Luxi Mono Regular} % muss aktiviert werden,
                                % falls das Paket installiert ist
\else
  %---- Eingabezeichensatz --------------------------------------------
  \usepackage[utf8]{inputenc}   % Eingabe deutscher Umlaute
                                % Unix/Linux: utf8
                                % Unix/Linux: latin1 (alt)
                                % Windows: cp1250
  %---- Grundfont -----------------------------------------------------
  \usepackage[T1]{fontenc}      % ec-Fonts
  \usepackage{lmodern}          % wg. der lm-Fonts (keine bitmap-Fonts!)
\fi

%---- Sprachauswahl ---------------------------------------------------
\usepackage{babel}              % um deutsche Bezeichnungen benutzen
                                % zu können, es wird der Parameter aus
                                % dem Kopf des Dokumentes benutzt.

%---- Verwaltung der Bibliographie, muss nach babel geladen werden ----
                                % Verwaltung der
                                % Bibliographie durch
\usepackage[backend=biber,      % Biber und biblatex
            autolang=other,     % Trennung gemäß der mit
                                % babel gesetzten Sprache
            style=alphabetic,   % Verweise ähnlich zu
                                % alpha.bst: XXX00
            citestyle=alphabetic, % mehrere Titel eines
                                % Autors werden XXX00a,
                                % XXX00b, ... zitiert
            giveninits=false,   % Vornamen werden nicht
                                % abgekürzt
            ]{biblatex}
\usepackage[babel,german=quotes]{csquotes} % Titel werden
                                % in deutsche Gänsefüßchen
                                % gesetzt
\addbibresource{../../shared/latex_course.bib}   % muss mit
                                % .bib-Datenbanken gefüllt werden
\ifluatex\else
  \usepackage{babelbib}         % fuer eine dazu passende Bibliographie,
                                % luatex kennt seine eigene
                                % Bibliographieverwaltung
\fi
\defbibheading{bibliography}[Literaturverzeichnis]{\chapter*{#1}}

%---- Sonstiges ------------------------------------------------------
% \PassOptionsToPackage{debugshow,final}{graphicx} % bei Bedarf zu
                                % aktivieren
\usepackage{graphicx}           % Vorbereitung der Graphiken
% \graphicspath{{...}{...}}     % muss mit entsprechenden Pfaden
                                % gefüllt werden, hier
\graphicspath{%                 % weitere Pfade können nach dem
  {../../shared/figures}%       % gleichen Muster hinzugefügt werden
}
\usepackage{listings}           % zur Darstellung von Code aller Art
\lstset{                        % Einstellungen des listings-Paketes
  language=tex,                 % es wird TeX\LaTeX-Code formatiert
  style=colored,                % der Hintergrund und spezielle Befehle
                                % werden farbig dargestellt,
  escapechar=@,                 % mit dem @-Zeichen gibt es die
                                % Möglichkeit,
                                % "ausgezeichneten" Text zu zeigen
}
%\usepackage{isodate}            % wg. \printdate{date}
\usepackage{booktabs}           % wg. \toprule, ...
                                % für typografisch korrekte Linien
\usepackage{moreverb}           % zum Schreiben von Texten in externe
                                % Dateien
\usepackage{tikz}               % TikZ ist kein Zeichenprogramm
\usetikzlibrary{trees}          % Produzieren von Baumstrukturen

\usepackage{pgfplots}           % Darstellung von Funktionen
\pgfplotsset{width=7cm,compat=1.18} % Standardbreite und Kompatibilitätslevel

\usepackage{amsmath}            % für die einfachere Eingabe math. Gleichungen,
                                % muss/sollte als Standard genutzt werden
\usepackage{rotating}           % für die nicht horizontale Textausrichtung
\usepackage{xspace}             % \wg. \xspace: fügt ein Leerzeichen an einen
                                % Text an, außer der Text ist von einem
                                % Punktuationszeichen gefolgt

                                % die nächsten Pakete dienen der Einstellung
                                % von Kopf- und Fußzeile
\usepackage[useregional]{datetime2}  % das Paket zur Darstellung von Tag und
                                % Uhrzeit in einem Format, das im deutschen
                                % Sprachbereich üblich ist
%\renewcommand{\dateseparator}{.}% das übliche Trennzeichen
\newsavebox{\NameAndDate}       % Name und Datum werden nur einmal bestimmt
                                % und dann zum späteren Gebrauch gespeichert
\usepackage[headsepline, footsepline]{scrlayer-scrpage} % das ist das
                                % eigentliche Paket, wir wollen Kopf- und
                                % Fußzeile durch Linien abtrennen

\pagestyle{scrheadings}         % das schaltet das neue Seitenlayout ein
\setkomafont{pagehead}{\normalfont\bfseries} % die Kopfzeile fett
\ifoot{\usebox{\NameAndDate}}   % Name und Datum innen auf der Seite
\AtBeginDocument{%              % wenn \begin{document} erreicht wird,
  \sbox{\NameAndDate}{%
    \footnotesize\textsf{%
      N.~N.: Meine eigene Probearbeit, \today--\DTMcurrenttime%
    }
  }
}

\usepackage{ifthen}             % zur Unterstützung logischer Bedingungen

%---- Bezuege --------------------------------------------------------
% gemäß der cleveref Dokumentation müssen diese Pakete als letzte
% geladen werden
\usepackage{varioref}           % Voraussetzung für cleveref
\usepackage[ngerman]{cleveref}  % deutschsprachige Bezuege, nach babel
                                % zu laden

%---- Einstellungen ---------------------------------------------------
\setcounter{secnumdepth}{7}     % Anzeige der Gliederungsstufen bis
                                % hinunter auf Ebene 7
\setcounter{tocdepth}{7}        % Inhaltsverzeichnis zeigt
                                % Gliederungsstufen bis
                                % hinunter auf Ebene 7, gebraucht wird
                                % das höchstens für technische
                                % Dokumentationen
\newlength{\myWidth}            % universelle Längenangabe, die
                                % jeweils passen gesetzt wird

%---- Eigene Definitionen ---------------------------------------------
\author{Jobst Hoffmann}
\title{Ein erstes \LaTeX-Dokument mit der Dokumentklasse
\texttt{scrartcl}}       % \texttt{<Inhalt>} formatiert <Inhalt>
                         % mit einer dicktengleichen Schriftart

% Hier werden nun eigene Befehle definiert, eine genaue Anleitung
% zur Definition eigener Befehle folgt später:
% \cs:    command sequence, das Argument wird als Befehl formatiert
\NewDocumentCommand{\cs}{m}{\texttt{\char92#1}}
% \farg:  fixed argument, das Argument wird als festes Argument eines
%        Befehls formatiert
\NewDocumentCommand{\farg}{m}
{%
  {\texttt{\{#1\}}}%
}
% \farg:  fixed optional argument, das Argument wird als optionales
%         Argument eines Befehls formatiert
\NewDocumentCommand{\foarg}{m}
{%
  {\texttt{[#1]}}%
}
% \marg:  mandatory argument, das Argument wird als verpflichtendes,
%         aber frei zu wählendes Argument eines Befehls formatiert
\NewDocumentCommand{\marg}{m}
{%
  {\texttt{\{}\(\langle\)\textit{#1}\(\rangle\)\texttt{\}}}%
}
% \meta:  das Argument wird als Beschreibung eines Objektes benutzt
\NewDocumentCommand{\meta}{m}
{%
  {\ensuremath{\langle}\textit{#1}\ensuremath{\rangle}}%
}
% \oarg:  optional argument, das Argument wird als optionales,
%         frei zu wählendes Argument eines Befehls formatiert
\NewDocumentCommand{\oarg}{m}
{%
  {\texttt{[}\(\langle\)\textit{#1}\(\rangle\)\texttt{]}}%
}
% \Cee: der Buchstabe C als Name der Programmiersprache,
%       \xspace sorgt dafür, dass das Kommando bei Bedarf
%       mit einem Leerzeichen ergänzt wird
\NewDocumentCommand{\Cee}{}{\textsc{C}\xspace}
     % ... allgemein
%23456789012345678901234567890123456789012345678901234567890123456789012
\wlog{This is lttr-cfg, the common configuration file for
   letters (JHf)}

% Wichtige Voreinstellungen
\usepackage{url}                % Paket zum Formatieren von URLs
\DeclareUrlCommand\email{\urlstyle{tt}}
\usepackage{phonenumbers}

% Blocksatz sieht zunächst gut aus, ist aber auf die Dauer langweilig und
% ermüdend, was gerade bei Anschreiben einer Bewerbung vermieden werden muss.
% Daher
% 1. merke die Originalkommandos der letter-Umgebung für späteren Zugriff
\let\letterorig=\letter
\let\endletterorig=\endletter
% 2. definiere die letter-Umgebung neu. Da diese Definition in einer 
%    Konfigurationsdatei steht, wirkt sie sich auf alle Briefe aus,
%    die sich dieser Konfiguration bedienen. 
%    Die Syntax ist zu beachten:
%    \renewenvironment{envname}[args]{begdef}{enddef}
%    Da die letter-Umgebung die Adresse als erstes Argument hat, muss dieses
%    von der neuen an die alte Umgebung weitergegeben werden.
\renewenvironment{letter}[1]{%
    \begin{letterorig}{#1}
        \parskip5pt %
        \flushleft
}{%
    \end{letterorig}%
}

% für das Aussehen des Briefbogens
\KOMAoptions{%
  foldmarks=true,               % Faltmarken werden gesetzt
  foldmarks=HP,                 % H: alle horizontalen FM,
                                % P: Marken für Locher (puncher)
  fromlogo=true,                % ermöglicht das Setzen eines Logos
  fromphone=false,              % eigene Telefonnummer und
  fromemail=true,               % email-Adresse
                                % false unterdrückt die entsprechende
                                % Ausgabe
  subject=titled
}

% für die Absenderangaben
\setkomavar{fromname}{Prof.\,Dr.~Jobst Hoffmann}
\setkomavar{signature}{Prof.\,Dr.~Jobst~Hoffmann} % \, sorgt für einen
                                % kleinen Zwischenraum, ~ ist der 
                                % Standardzwischenraum, der nicht 
                                % umgebrochen werden kann 
\setkomavar{backaddress}{%
	Prof. Dr. J. Hoffmann\\
	Heinrich-Mußmann-Str. 1\\
	52428 Jülich
}
\setkomavar{fromaddress}{%
	Heinrich-Mußmann-Str. 1\\
	52428 Jülich
	}
%\setkomavar{fromphone}{\phonenumber[foreign]{02416009}[53179]} % nicht
                                 % mehr gültig
\setkomavar{fromemail}{\email{j.hoffmann@fh-aachen}}
    % ... spezifisch für einen Brief

%\setkomavar{yourmail}{15.10.2022}
\setkomavar{yourmail}{\printdate{2022-10-15}} % setzt Paket isodate
                                 % voraus, Datumsangabe im ISO-Format
                                 % ISO: International Standardization
                                 % Organisation, ISO-Standards sind in
                                 % der IT-Branche extrem wichtig, um
                                 % einwandfreien Informationsaustausch
                                 % zu gewährleisten: YYYY-MM-DD als
                                 % Vorlage 


%\setkomavar{title}{Dankeschön}  % zu aufdringlich
% \setkomavar{subject}{Dankeschön}
\setkomavar{subject}{Rechnung}
\setkomavar{frombank}{Meine Bank\\IBAN: DE12 1234 1234 1234 1234 00}
\setkomavar{firstfoot}{%
  \parbox[b]{\linewidth}{%
    \centering\def\\{, }\usekomavar{frombank}% hier wird lokal
                                 % der Zeilenumbruch umdefiniert
                                 % in ein Komma
  }%
}
\begin{document}

\begin{letter}
  {%
   An die \\                    % die Adresse der Empfängerin/
   Fachhochschule Aachen \\     % des Empfängers, jede Zeile wird mit
   Campus Jülich \\             % \\ getrennt
   Prüfungsamt \\
   Heinrich-Mußmann-Str. 1 \\[3mm] % [...] gibt zusätzlichen Leerraum
   52428 Jülich                 % an
  }

  \opening{Sehr geehrte Frau Surma,}
  
%   vielen Dank für die schnelle Fertigstellung der Bescheinigung meiner
%   bestandenen Prüfung.
  vielen Dank für den Auftrag, den ich sehr zu schätzen weiß. In der Anlage finden Sie die zugehörige Rechnung, den Betrag von 1,00€ überweisen Sie bitte auf das unten angegebene Konto. Wenn es weiteren Bedarf an meinen Tätigkeiten
  und Produkten gibt, lassen Sie es mich bitte wissen; ich werde Ihnen stets
  ein auf Ihre Anforderungen zugeschnittenes Angebot zukommen lassen.
  
  Beachten Sie bitte, dass dieser Brief nicht im Blocksatz formatiert worden
  ist. Ebenso hat dieser Abschnitt einen etwas größeren Abstand 
  als eine Zeile vom vorigen Abschnitt. Dies habe ich so eingestellt,
  damit Sie vom Lesen dieses Briefes nicht gelangweilt und ermüdet werden,
  und damit neue Abschnitte sofort erkennbar sind und der Brief somit besser
  lesbar ist. 
  
  \closing{Mit freundlichen Grüßen}
  \encl{Rechnung}
\end{letter}
  
\end{document}
