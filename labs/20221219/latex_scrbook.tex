% Leslie Lamport hat LaTeX entwickelt, um dem Anwender das Schreiben
% spezieller Dokumente zu vereinfachen. Um den europäischen/deutschen
% Vorstellungen eines Layouts zu entsprechen, hat Markus Kohm analoge
% Dokumentklassen entwickelt (Koma-Klassen)

% Diese Dokumente sind
% im Original    extern              Koma       extern
% 1. article                         scrartcl
% 2. report                          scrreprt
% 3. book                            scrbook
% 4. letter                          scrlttr2
% 5. slide                                      beamer
% 6.             poster                         tikzposter
% 7.                                            komacv bzw. europecv
% Das Layout ist im Original amerikanischen Vorstellungen ensprechend!

% Der Aufbau ist immer

% Kopf des Dokumentes
% ===================
\documentclass[ngerman]{scrbook}      % in eckigen Klammern stehen 
                                      % optionale Angaben, hier: ngerman
                                      % - es werden Trennmuster nach
                                      %   _n_euer deutscher Recht-
                                      %   schreibung geladen
                                      % - wenn das Zusatzpaket babel
                                      %   geladen ist, kann der Autor
                                      %   länderspezifische
                                      %   Befehle, beispielsweise für
                                      %   Anführungszeiche direkt
                                      %   eingeben

                                      % typographische Regeln:
                                      % in der scrbook-Klasse beginnt
                                      % jedes Kapitel auf einer
                                      % rechten, 
                                      % ungeraden Seite, leere Seiten
                                      % erhalten keine Seitennummer, 
                                      % ebensowenig Titelseiten, es
                                      % werden alle(!!!) Seiten intern
                                      % nummeriert, die Titelseite 
                                      % hat die Nummer 1, die
                                      % Seitennummerierung erfolgt
                                      % unten außen (rechte Seite
                                      % rechts, linke Seite links)
                                      
                                      % in der scrbook-Klasse werden
                                      % frontispiz, ...
                                      % unterschieden
                                      % \frontmatter - der Inhalt vor
                                      % dem ersten Kapitel
                                      % \mainmatter - der "Hauptinhalt"
                                      %    Kapitel 1 - Kapitel <n>
                                      % \backmatter - alles, was danach
                                      % kommt: Glossar (optional),
                                      % Literaturverzeichnis,
                                      % Stichwortverzeichnis bzw. Index
                                      % (optional),
                                      % die Reihenfolge ergibt sich aus
                                      % der Wichtigkeit für den Leser,
                                      % etwas, was weiterhinten steht,
                                      % kann direkt angeblättert werden
                                      
                                      
% Einstellungen
% =============

% Da die Einstellungen sich von Dokument zu Dokument kaum verändern,
% ist es sinnvoll, diese an einer zentralen Stelle abzulegen und von 
% dort zu laden. Dies geschieht völlig transparent mit dem Befehl
% \input{<Pfad>}, dem als Argument eine Datei mit (relativem) Zugriffs-
% pfad übergeben wird
\input{../../shared/twp-cfg}

% merke: ein späterer Titel überschreibt einen früher gesetzten Titel,
% in der scrreprt-Dokumentenklasse erscheint der Titel auf einer
% eigenen Seite
\title{Ein erstes \LaTeX-Dokument mit der Dokumentklasse \texttt{scrbook}}       % \texttt{<Inhalt>} formatiert <Inhalt>
                         % mit einer dicktengleichen Schriftart
                         % im Vergleich zur ansonsten eingesetzten
                         % Proportionalschrift

% Rumpf des Dokumentes
\begin{document}
  \frontmatter                        % damit beginnt das Buch, die 
                                      % Seitennummerierung erfolgt mit
                                      % römischen Ziffern i, ii, iii,
                                      % ...
  
  \maketitle                          % die automatisch erzeugte
                                      % Titelseite, dies erfordert
                                      % die Angabe des Autors und des
                                      % Titels in den Einstellungen
           
  % Die Zusammenfassung erscheint auf einer eigenen Seite. Da es in
  % scrbook keine abstract-Umgebung gibt, muss sie ersetzt werden
  % durch
  \chapter*{Vorwort}                  % die abstract-Umgebung gibt es
                                      % nur in der scrartcl-Klasse!
                                      % Der Stern unterdrückt die
                                      % Nummerierung der zugehörigen
                                      % Überschrift
    Hier steht eine kurze Zusammenfassung des Inhaltes, es geht hier um den Aufbau eines Artikels, insbesondere um die geeignete Gliederung.

  \tableofcontents               

  % die Gliederung erfolgt in bis zu 7 Ebenen, abhängig von der Klasse
  %   Ebene     Klasse                   Befehl (einfachen Form)
  %   0.        scrbook/scrreprt         \chapter{<Titel>}
  %   1.        alle                     \section{<Titel>}
  %   2.        alle                     \subsection{<Titel>}
  %   3.        alle                     \subsubsection{<Titel>}
  %   4.        alle                     \paragraph{<Titel>}
  %   5.        alle                     \subparagraph{<Titel>}
  %   6.        alle, nicht im Original  \subsubparagraph{<Titel>}
  
  \mainmatter            % hier folgt der Inhalt, die Seitennummerierung
                         % erfolgt mit arabischen Ziffern, in einer
                         % Bibliographie wird angegeben vii+23 Seiten
  
  \chapter{Einleitung}   % Einleitung, Hauptteil und Schluss
                         % dürfen niemals in der fertigen Arbeit
                         % stehen, da muss die Autorin bzw. der Autor
                         % sich etwas Passendes einfallen lassen.

  Hier steht der Inhalt des Dokumentes, weiteres siehe im n\"achsten
  Beispiel. Der erste Absatz wird ohne Einzug formatiert, wenn er einer \"Uberschrift folgt.
  
  Weitere Absätze bekommen einen Einzug, wie hier deutlich zu sehen ist.
  Das soll an dieser Stelle reichen.
  
  \chapter{Hauptteil}
  
  Text des Hauptteils mit Unterabschnitten
  
  % eine Kurzform der Überschrift, die für das Inhaltsverzeichnis
  % geeignet ist erzeugt man mit einem optionalen Argument
  \section[Erster Unterabschnitt]{Erster Unterabschnitt zu
dem Hauptteil, der in der Zukunft einen wesentlich aussagekräfigeren
Titel bekommen muss}

  % die Nummerierung wird automatisch korekt erzeugt, wenn die
  % Gliederung korrekt ist, eine fehlerhafte Gliederung wird durch 0.
  % in der Überschrift angezeigt. Für die Hausaufgaben ist eine
  % fehlerhafte Nummerierung ein Ausschlusskriterium
  \subsection{Weitere Untergliederung Teil 1}
  \subsection{Weitere Untergliederung Teil 2}
  \subsection{Weitere Untergliederung Teil 3}
  \subsection{Weitere Untergliederung Teil 4}

  \section[Zweiter Unterabschnitt]{Zweiter Unterabschnitt zu
dem Hauptteil, der in der Zukunft einen wesentlich aussagekräfigeren
Titel bekommen muss}
  
  \section[Dritter Unterabschnitt]{Dritter Unterabschnitt zu
dem Hauptteil, der in der Zukunft einen wesentlich aussagekräfigeren
Titel bekommen muss}
  
  \chapter{Schluss}
  
  \appendix                     % hiermit wird der Anhang eingeleitet,
                                % die Kapitelzählung beginnt neu mit
                                % Großbuchstaben A., B., ... 
  
  \chapter{Programmbeispiele}
  
  \backmatter                   % z. Zt. noch leer
\end{document}
