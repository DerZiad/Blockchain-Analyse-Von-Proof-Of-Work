% Leslie Lamport hat LaTeX entwickelt, um dem Anwender das Schreiben
% spezieller Dokumente zu vereinfachen. Um den europäischen/deutschen
% Vorstellungen eines Layouts zu entsprechen, hat Markus Kohm analoge
% Dokumentklassen entwickelt (Koma-Klassen)

% Diese Dokumente sind
% im Original    extern              Koma       extern
% 1. article                         scrartcl
% 2. report                          scrreprt
% 3. book                            scrbook
% 4. letter                          scrlttr2
% 5. slide                                      beamer
% 6.             poster                         tikzposter
% 7.                                            komacv bzw. europecv
% Das Layout ist im Original amerikanischen Vorstellungen ensprechend!

% Der Aufbau ist immer

% Kopf des Dokumentes
% ===================
\documentclass[ngerman]{scrbook}      % in eckigen Klammern stehen 
                                      % optionale Angaben, hier: ngerman
                                      % - es werden Trennmuster nach
                                      %   _n_euer deutscher Recht-
                                      %   schreibung geladen
                                      % - wenn das Zusatzpaket babel
                                      %   geladen ist, kann der Autor
                                      %   länderspezifische
                                      %   Befehle, beispielsweise für
                                      %   Anführungszeiche direkt
                                      %   eingeben

                                      % typographische Regeln:
                                      % in der scrbook-Klasse beginnt
                                      % jedes Kapitel auf einer
                                      % rechten, 
                                      % ungeraden Seite, leere Seiten
                                      % erhalten keine Seitennummer, 
                                      % ebensowenig Titelseiten, es
                                      % werden alle(!!!) Seiten intern
                                      % nummeriert, die Titelseite 
                                      % hat die Nummer 1, die
                                      % Seitennummerierung erfolgt
                                      % unten außen (rechte Seite
                                      % rechts, linke Seite links)
                                      
                                      % in der scrbook-Klasse werden
                                      % frontispiz, ...
                                      % unterschieden
                                      % \frontmatter - der Inhalt vor
                                      % dem ersten Kapitel
                                      % \mainmatter - der "Hauptinhalt"
                                      %    Kapitel 1 - Kapitel <n>
                                      % \backmatter - alles, was danach
                                      % kommt: Glossar (optional),
                                      % Literaturverzeichnis,
                                      % Stichwortverzeichnis bzw. Index
                                      % (optional),
                                      % die Reihenfolge ergibt sich aus
                                      % der Wichtigkeit für den Leser,
                                      % etwas, was weiterhinten steht,
                                      % kann direkt angeblättert werden
                                      
                                      
% Einstellungen
% =============

% Da die Einstellungen sich von Dokument zu Dokument kaum verändern,
% ist es sinnvoll, diese an einer zentralen Stelle abzulegen und von 
% dort zu laden. Dies geschieht völlig transparent mit dem Befehl
% \input{<Pfad>}, dem als Argument eine Datei mit (relativem) Zugriffs-
% pfad übergeben wird
%23456789012345678901234567890123456789012345678901234567890123456789012
\wlog{This is twp-cfg, the common configuration file all documents
  (JHf)}

% Einstellungen werden in LaTeX vor allen Dingen durch das Einbinden
% von Paketen vorgenommen:
% \usepackage[parameter]{package}

\usepackage{iftex}              % Test des Formatierers, durch \if...
                                % können Anpassungen an den Prozessor
                                % vorgenommen werden
\ifluatex                       % wenn mit dem neuen TeX-Prozessor,
                                % der mehr Fontarten unterstützt,
                                % gearbeitet wird:
                                % - pk-Fonts (Pixelfonts, nicht
                                %   skalierbar, alt)
                                % - pfb-Fonts (Fonts, die durch Kurven
                                %   beschrieben werden, skalierbar,
                                %   entwickelt von Adobe->Acrobat
                                %   Reader)
                                % - otf/ttf-Fonts (Fonts, die durch
                                %   Kurven beschrieben werden,
                                %   skalierbar, entwickelt von Apple
                                %   und Microsoft, um Adobe-Lizenzen
                                %   zu sparen)
  %---- Eingabezeichensatz --------------------------------------------
                                % luatex unterstützt utf-8, also keine
                                % Festlegung des Eingabezeichensatzes
                                % erforderlich
  %---- Grundfont -----------------------------------------------------
  \usepackage{fontspec}         % Festlegen der Fontverwaltung für
                                % LuaTeX.
  \defaultfontfeatures{Renderer=Basic, Ligatures=TeX}
  \fontspec{Latin Modern Roman}
  \setmonofont[Scale=0.85]{Luxi Mono Regular} % muss aktiviert werden,
                                % falls das Paket installiert ist
\else
  %---- Eingabezeichensatz --------------------------------------------
  \usepackage[utf8]{inputenc}   % Eingabe deutscher Umlaute
                                % Unix/Linux: utf8
                                % Unix/Linux: latin1 (alt)
                                % Windows: cp1250
  %---- Grundfont -----------------------------------------------------
  \usepackage[T1]{fontenc}      % ec-Fonts
  \usepackage{lmodern}          % wg. der lm-Fonts (keine bitmap-Fonts!)
\fi

%---- Sprachauswahl ---------------------------------------------------
\usepackage{babel}              % um deutsche Bezeichnungen benutzen
                                % zu können, es wird der Parameter aus
                                % dem Kopf des Dokumentes benutzt.

%---- Verwaltung der Bibliographie, muss nach babel geladen werden ----
                                % Verwaltung der
                                % Bibliographie durch
\usepackage[backend=biber,      % Biber und biblatex
            autolang=other,     % Trennung gemäß der mit
                                % babel gesetzten Sprache
            style=alphabetic,   % Verweise ähnlich zu
                                % alpha.bst: XXX00
            citestyle=alphabetic, % mehrere Titel eines
                                % Autors werden XXX00a,
                                % XXX00b, ... zitiert
            giveninits=false,   % Vornamen werden nicht
                                % abgekürzt
            ]{biblatex}
\usepackage[babel,german=quotes]{csquotes} % Titel werden
                                % in deutsche Gänsefüßchen
                                % gesetzt
\addbibresource{../../shared/latex_course.bib}   % muss mit
                                % .bib-Datenbanken gefüllt werden
\ifluatex\else
  \usepackage{babelbib}         % fuer eine dazu passende Bibliographie,
                                % luatex kennt seine eigene
                                % Bibliographieverwaltung
\fi
\defbibheading{bibliography}[Literaturverzeichnis]{\chapter*{#1}}

%---- Sonstiges ------------------------------------------------------
% \PassOptionsToPackage{debugshow,final}{graphicx} % bei Bedarf zu
                                % aktivieren
\usepackage{graphicx}           % Vorbereitung der Graphiken
% \graphicspath{{...}{...}}     % muss mit entsprechenden Pfaden
                                % gefüllt werden, hier
\graphicspath{%                 % weitere Pfade können nach dem
  {../../shared/figures}%       % gleichen Muster hinzugefügt werden
}
\usepackage{listings}           % zur Darstellung von Code aller Art
\lstset{                        % Einstellungen des listings-Paketes
  language=tex,                 % es wird TeX\LaTeX-Code formatiert
  style=colored,                % der Hintergrund und spezielle Befehle
                                % werden farbig dargestellt,
  escapechar=@,                 % mit dem @-Zeichen gibt es die
                                % Möglichkeit,
                                % "ausgezeichneten" Text zu zeigen
}
%\usepackage{isodate}            % wg. \printdate{date}
\usepackage{booktabs}           % wg. \toprule, ...
                                % für typografisch korrekte Linien
\usepackage{moreverb}           % zum Schreiben von Texten in externe
                                % Dateien
\usepackage{tikz}               % TikZ ist kein Zeichenprogramm
\usetikzlibrary{trees}          % Produzieren von Baumstrukturen

\usepackage{pgfplots}           % Darstellung von Funktionen
\pgfplotsset{width=7cm,compat=1.18} % Standardbreite und Kompatibilitätslevel

\usepackage{amsmath}            % für die einfachere Eingabe math. Gleichungen,
                                % muss/sollte als Standard genutzt werden
\usepackage{rotating}           % für die nicht horizontale Textausrichtung
\usepackage{xspace}             % \wg. \xspace: fügt ein Leerzeichen an einen
                                % Text an, außer der Text ist von einem
                                % Punktuationszeichen gefolgt

                                % die nächsten Pakete dienen der Einstellung
                                % von Kopf- und Fußzeile
\usepackage[useregional]{datetime2}  % das Paket zur Darstellung von Tag und
                                % Uhrzeit in einem Format, das im deutschen
                                % Sprachbereich üblich ist
%\renewcommand{\dateseparator}{.}% das übliche Trennzeichen
\newsavebox{\NameAndDate}       % Name und Datum werden nur einmal bestimmt
                                % und dann zum späteren Gebrauch gespeichert
\usepackage[headsepline, footsepline]{scrlayer-scrpage} % das ist das
                                % eigentliche Paket, wir wollen Kopf- und
                                % Fußzeile durch Linien abtrennen

\pagestyle{scrheadings}         % das schaltet das neue Seitenlayout ein
\setkomafont{pagehead}{\normalfont\bfseries} % die Kopfzeile fett
\ifoot{\usebox{\NameAndDate}}   % Name und Datum innen auf der Seite
\AtBeginDocument{%              % wenn \begin{document} erreicht wird,
  \sbox{\NameAndDate}{%
    \footnotesize\textsf{%
      N.~N.: Meine eigene Probearbeit, \today--\DTMcurrenttime%
    }
  }
}

\usepackage{ifthen}             % zur Unterstützung logischer Bedingungen

%---- Bezuege --------------------------------------------------------
% gemäß der cleveref Dokumentation müssen diese Pakete als letzte
% geladen werden
\usepackage{varioref}           % Voraussetzung für cleveref
\usepackage[ngerman]{cleveref}  % deutschsprachige Bezuege, nach babel
                                % zu laden

%---- Einstellungen ---------------------------------------------------
\setcounter{secnumdepth}{7}     % Anzeige der Gliederungsstufen bis
                                % hinunter auf Ebene 7
\setcounter{tocdepth}{7}        % Inhaltsverzeichnis zeigt
                                % Gliederungsstufen bis
                                % hinunter auf Ebene 7, gebraucht wird
                                % das höchstens für technische
                                % Dokumentationen
\newlength{\myWidth}            % universelle Längenangabe, die
                                % jeweils passen gesetzt wird

%---- Eigene Definitionen ---------------------------------------------
\author{Jobst Hoffmann}
\title{Ein erstes \LaTeX-Dokument mit der Dokumentklasse
\texttt{scrartcl}}       % \texttt{<Inhalt>} formatiert <Inhalt>
                         % mit einer dicktengleichen Schriftart

% Hier werden nun eigene Befehle definiert, eine genaue Anleitung
% zur Definition eigener Befehle folgt später:
% \cs:    command sequence, das Argument wird als Befehl formatiert
\NewDocumentCommand{\cs}{m}{\texttt{\char92#1}}
% \farg:  fixed argument, das Argument wird als festes Argument eines
%        Befehls formatiert
\NewDocumentCommand{\farg}{m}
{%
  {\texttt{\{#1\}}}%
}
% \farg:  fixed optional argument, das Argument wird als optionales
%         Argument eines Befehls formatiert
\NewDocumentCommand{\foarg}{m}
{%
  {\texttt{[#1]}}%
}
% \marg:  mandatory argument, das Argument wird als verpflichtendes,
%         aber frei zu wählendes Argument eines Befehls formatiert
\NewDocumentCommand{\marg}{m}
{%
  {\texttt{\{}\(\langle\)\textit{#1}\(\rangle\)\texttt{\}}}%
}
% \meta:  das Argument wird als Beschreibung eines Objektes benutzt
\NewDocumentCommand{\meta}{m}
{%
  {\ensuremath{\langle}\textit{#1}\ensuremath{\rangle}}%
}
% \oarg:  optional argument, das Argument wird als optionales,
%         frei zu wählendes Argument eines Befehls formatiert
\NewDocumentCommand{\oarg}{m}
{%
  {\texttt{[}\(\langle\)\textit{#1}\(\rangle\)\texttt{]}}%
}
% \Cee: der Buchstabe C als Name der Programmiersprache,
%       \xspace sorgt dafür, dass das Kommando bei Bedarf
%       mit einem Leerzeichen ergänzt wird
\NewDocumentCommand{\Cee}{}{\textsc{C}\xspace}


% merke: ein späterer Titel überschreibt einen früher gesetzten Titel,
% in der scrreprt-Dokumentenklasse erscheint der Titel auf einer
% eigenen Seite
\title{Ein erstes \LaTeX-Dokument mit der Dokumentklasse \texttt{scrbook}}       % \texttt{<Inhalt>} formatiert <Inhalt>
                         % mit einer dicktengleichen Schriftart
                         % im Vergleich zur ansonsten eingesetzten
                         % Proportionalschrift

% Rumpf des Dokumentes
\begin{document}
  \frontmatter                        % damit beginnt das Buch, die 
                                      % Seitennummerierung erfolgt mit
                                      % römischen Ziffern i, ii, iii,
                                      % ...
  
  \maketitle                          % die automatisch erzeugte
                                      % Titelseite, dies erfordert
                                      % die Angabe des Autors und des
                                      % Titels in den Einstellungen
           
  % Die Zusammenfassung erscheint auf einer eigenen Seite. Da es in
  % scrbook keine abstract-Umgebung gibt, muss sie ersetzt werden
  % durch
  \chapter*{Vorwort}                  % die abstract-Umgebung gibt es
                                      % nur in der scrartcl-Klasse!
                                      % Der Stern unterdrückt die
                                      % Nummerierung der zugehörigen
                                      % Überschrift
    Hier steht eine kurze Zusammenfassung des Inhaltes, es geht hier um den Aufbau eines Artikels, insbesondere um die geeignete Gliederung.

  \tableofcontents               

  % die Gliederung erfolgt in bis zu 7 Ebenen, abhängig von der Klasse
  %   Ebene     Klasse                   Befehl (einfachen Form)
  %   0.        scrbook/scrreprt         \chapter{<Titel>}
  %   1.        alle                     \section{<Titel>}
  %   2.        alle                     \subsection{<Titel>}
  %   3.        alle                     \subsubsection{<Titel>}
  %   4.        alle                     \paragraph{<Titel>}
  %   5.        alle                     \subparagraph{<Titel>}
  %   6.        alle, nicht im Original  \subsubparagraph{<Titel>}
  
  \mainmatter            % hier folgt der Inhalt, die Seitennummerierung
                         % erfolgt mit arabischen Ziffern, in einer
                         % Bibliographie wird angegeben vii+23 Seiten
  
  \chapter{Einleitung}   % Einleitung, Hauptteil und Schluss
                         % dürfen niemals in der fertigen Arbeit
                         % stehen, da muss die Autorin bzw. der Autor
                         % sich etwas Passendes einfallen lassen.

  Hier steht der Inhalt des Dokumentes, weiteres siehe im n\"achsten
  Beispiel. Der erste Absatz wird ohne Einzug formatiert, wenn er einer \"Uberschrift folgt.
  
  Weitere Absätze bekommen einen Einzug, wie hier deutlich zu sehen ist.
  Das soll an dieser Stelle reichen.
  
  \chapter{Hauptteil}
  
  Text des Hauptteils mit Unterabschnitten
  
  % eine Kurzform der Überschrift, die für das Inhaltsverzeichnis
  % geeignet ist erzeugt man mit einem optionalen Argument
  \section[Erster Unterabschnitt]{Erster Unterabschnitt zu
dem Hauptteil, der in der Zukunft einen wesentlich aussagekräfigeren
Titel bekommen muss}

  % die Nummerierung wird automatisch korekt erzeugt, wenn die
  % Gliederung korrekt ist, eine fehlerhafte Gliederung wird durch 0.
  % in der Überschrift angezeigt. Für die Hausaufgaben ist eine
  % fehlerhafte Nummerierung ein Ausschlusskriterium
  \subsection{Weitere Untergliederung Teil 1}
  \subsection{Weitere Untergliederung Teil 2}
  \subsection{Weitere Untergliederung Teil 3}
  \subsection{Weitere Untergliederung Teil 4}

  \section[Zweiter Unterabschnitt]{Zweiter Unterabschnitt zu
dem Hauptteil, der in der Zukunft einen wesentlich aussagekräfigeren
Titel bekommen muss}
  
  \section[Dritter Unterabschnitt]{Dritter Unterabschnitt zu
dem Hauptteil, der in der Zukunft einen wesentlich aussagekräfigeren
Titel bekommen muss}
  
  \chapter{Schluss}
  
  \appendix                     % hiermit wird der Anhang eingeleitet,
                                % die Kapitelzählung beginnt neu mit
                                % Großbuchstaben A., B., ... 
  
  \chapter{Programmbeispiele}
  
  \backmatter                   % z. Zt. noch leer
\end{document}
