% Kommentar wird eingeleitet durch ein %-Zeichen; alles, was dem
%-Zeichen bis zum Zeilenende folgt, ist Kommentar

% Text besteht aus Zeichen/Buchstaben: jedes Zeichen ist ein primitives
% TeX-Kommando. Die Bedeutung eines solchen Kommandos ist:
% - setze das Zeichen
% - im der aktuellen Größe
% - an die aktuelle Stelle
% - und verschiebe die aktuelle Stelle 
% - um die Breite des gesetzten Zeichens
% um für das nächste Zeichen bereit zu sein.

% Damit wir eine ganz lange Eingabezeile erzeugt. Diese Eingabezeile
% stellt einen Absatz dar. Ist das Absatz beendet (durch einen
% expliziten Befehl oder durch den impliziten Befehl "Leerzeile"), macht
% TeX aus dem Absatz eine Menge von Zeilen (im Original: breaking
% paragraphs into lines, der Algorithmus dazu ist von einem Assistenten
% von D.E. Knuth entwickelt worden).

Das ist ein erster Satz, der mit TeX formatiert werden soll.

Jetzt folgt mehr Text, der über eine Zeile hinausgeht, damit der
automatische Zeilenumbruch gezeigt werden kann. Dieser Text kann nun
beliebig lang sein, bei Bedarf auch über mehrere Seiten gehen. Dass
ein solch langer Text nicht gut zu lesen ist, ist klar, es wird deshalb
nun eine neuer Absatz begonnen.

Der neue Absatz wird durch den impliziten Befehl "Leerzeile" eingeletet.
Dem Leser wird der neue Absatz deutlich gemacht durch einen so genannten
Einzug: die erste Zeile eines Absatzes wird um eine feste Breite nach
rechts verschoben.

Zwei Dinge sollten noch vermerkt werden: TeX wird innehalb eines Textes
immer durch den Befehl \TeX{} dargestellt, Anf\"uhrungszeichen wie um 
Leerzeile werden nicht durch G\"ansef\"u\ss chen erzeugt, sondern mit 
speziellen Zeichenkombinationen: ``Leerzeile''.


% jedes TeX-Programm muss mit dem zugehörigen Ende-Befehl versehen
% werden

\bye

% Ein TeX-Befehl beginnt mit einem Gegenschrägstrich (backslash), er
% besteht aus beliebig vielen Buchstaben, ein Nicht-Buchstabe beendet
% den Befehl. Nicht-Buchstaben sind
% - Leerzeichen
% - geschweifte, eckige und runde Klammern, erstere dienen der
%   Darstellung von erforderlichen Parametern, die zweiten werden für
%   optionale Parameter benötigt
% - Ziffern (0, 1, ...) und Sonderzeichen (., ;, ...)

