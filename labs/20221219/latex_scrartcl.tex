% Leslie Lamport hat LaTeX entwickelt, um dem Anwender das Schreiben
% spezieller Dokumente zu vereinfachen. Um den europäischen/deutschen
% Vorstellungen eines Layouts zu entsprechen, hat Markus Kohm analoge
% Dokumentklassen entwickelt (Koma-Klassen)

% Diese Dokumente sind
% im Original    extern              Koma       extern
% 1. article                         scrartcl
% 2. report                          scrreprt
% 3. book                            scrbook
% 4. letter                          scrlttr2
% 5. slide                                      beamer
% 6.             poster                         tikzposter
% 7.                                            komacv bzw. europecv
% Das Layout ist im Original amerikanischen Vorstellungen ensprechend!

% Der Aufbau ist immer

% Kopf des Dokumentes
% ===================
\documentclass[ngerman]{scrartcl}     % in eckigen Klammern stehen 
                                      % optionale Angaben, hier: ngerman
                                      % - es werden Trennmuster nach
                                      %   _n_euer deutscher Recht-
                                      %   schreibung geladen
                                      % - wenn das Zusatzpaket babel
                                      %   geladen ist, kann der Autor
                                      %   länderspezifische
                                      %   Befehle, beispielsweise für
                                      %   Anführungszeiche direkt
                                      %   eingeben

% Einstellungen
% =============

% Da die Einstellungen sich von Dokument zu Dokument kaum verändern,
% ist es sinnvoll, diese an einer zentralen Stelle abzulegen und von 
% dort zu laden. Dies geschieht völlig transparent mit dem Befehl
% \input{<Pfad>}, dem als Argument eine Datei mit (relativem) Zugriffs-
% pfad übergeben wird
\input{../../shared/twp-cfg}

% Rumpf des Dokumentes
\begin{document}
  \maketitle                          % die automatisch erzeugte
                                      % Titelseite, dies erfordert
                                      % die Angabe des Autors und des
                                      % Titels in den Einstellungen
           
           
  \begin{abstract}                    % die abstract-Umgebung gibt es
                                      % nur in der scrartcl-Klasse!
    Hier steht eine kurze Zusammenfassung des Inhaltes, es geht hier um den Aufbau eines Artikels, insbesondere um die geeignete Gliederung.
  \end{abstract}                     

  \tableofcontents               

  % die Gliederung erfolgt in bis zu 7 Ebenen, abhängig von der Klasse
  %   Ebene     Klasse                   Befehl (einfachen Form)
  %   0.        scrbook/scrreprt         \chapter{<Titel>}
  %   1.        alle                     \section{<Titel>}
  %   2.        alle                     \subsection{<Titel>}
  %   3.        alle                     \subsubsection{<Titel>}
  %   4.        alle                     \paragraph{<Titel>}
  %   5.        alle                     \subparagraph{<Titel>}
  %   6.        alle, nicht im Original  \subsubparagraph{<Titel>}
  
  \section{Einleitung}   % Einleitung, Haauptteil und Schluss
                         % dürfen niemals in der fertigen Arbeit
                         % stehen, da muss die Autorin bzw. der Autor
                         % sich etwas Passendes einfallen lassen.

  Hier steht der Inhalt des Dokumentes, weiteres siehe im n\"achsten
  Beispiel. Der erste Absatz wird ohne Einzug formatiert, wenn er einer \"Uberschrift folgt.
  
  Weitere Absätze bekommen einen Einzug, wie hier deutlich zu sehen ist.
  Das soll an dieser Stelle reichen.
  
  \section{Hauptteil}
  
  Text des Hauptteils mit Unterabschnitten
  
  % eine Kurzform der Überschrift, die für das Inhaltsverzeichnis
  % geeignet ist erzeugt man mit einem optionalen Argument
  \subsection[Erster Unterabschnitt]{Erster Unterabschnitt zu
dem Hauptteil, der in der Zukunft einen wesentlich aussagekräfigeren
Titel bekommen muss}

  % die Nummerierung wird automatisch korekt erzeugt, wenn die
  % Gliederung korrekt ist, eine fehlerhafte Gliederung wird durch 0.
  % in der Überschrift angezeigt. Für die Hausaufgaben ist eine
  % fehlerhafte Nummerierung ein Ausschlusskriterium
  \subsubsection{Weitere Untergliederung Teil 1}
  \subsubsection{Weitere Untergliederung Teil 2}
  \subsubsection{Weitere Untergliederung Teil 3}
  \subsubsection{Weitere Untergliederung Teil 4}

  \subsection[Zweiter Unterabschnitt]{Zweiter Unterabschnitt zu
dem Hauptteil, der in der Zukunft einen wesentlich aussagekräfigeren
Titel bekommen muss}
  
  \subsection[Dritter Unterabschnitt]{Dritter Unterabschnitt zu
dem Hauptteil, der in der Zukunft einen wesentlich aussagekräfigeren
Titel bekommen muss}
  
  \section{Schluss}
  
\end{document}
